\title{Numerical Methods Notes}
\author{Harris Hancock}
\date{\today}
\documentclass[12pt]{article}

\usepackage{verbatim}
\usepackage{amsmath}
\usepackage{amsfonts}

\numberwithin{equation}{section}

\begin{document}
\maketitle

\section{Polynomial Synthetic Division}
Synthetic division is an algorithm to perform Euclidean division of two
polynomials. Given a \(k\)-degree dividend \(p\), an \(m\)-degree divisor
\(s\), \(m \leq k\), synthetic division finds the polynomials \(q\) and
\(r\) so that the following is true:
\begin{align*}
	p(x)& \in \mathbb{P}^k&
	q(x)& \in \mathbb{P}^{k-m} \\
	s(x)& \in \mathbb{P}^m&
	r(x)& \in \mathbb{P}^{m-1}
\end{align*}
\begin{equation}
	\frac{p(x)}{s(x)} = q(x) + \frac{r(x)}{s(x)}
\end{equation}

The formulas for the quotient and remainder are
\begin{align}
	\begin{split}
	q_{k-m+1}& \equiv q_{k-m+2} \equiv \dotsb \equiv 0 \\
	q_i& = \frac{1}{s_m}\left(p_{m+i}-\sum_{j=1}^m{s_{m-j}q_{i+j}}\right)
	\end{split}\\
	r_i& = p_{i}-\sum_{j=0}^{i}{s_{i-j}q_j}
\end{align}

Synthetic division is related to Horner's Method. To evaluate \(p(a)\), we can
take the remainder of \(\frac{p(x)}{(x-a)}\).
\begin{align*}
               	r(x) = r_0& = p(x)-(x-a)q(x) = p(a)
	\intertext{Substitute in the formula for \(r_0\), then again for \(q_0\).}
		          & = p_0-\sum_{j=0}^0{s_{0-j}q_j} \\
		          & = p_0-s_0q_0 \\
			  & = p_0-s_0\left(\frac{1}{s_1}\left(p_1-\sum_{j=1}^1{s_{1-j}q_{0+j}}\right)\right)
	\intertext{Since the divisor \(s(x) = x - a\) is a monic, 1-degree polynomial, then \(\frac{1}{s_1}=1\), and the sum will always only have a single iteration.}
			  & = p_0-s_0\left(p_1-s_0q_1\right) \\
			  & = p_0-s_0\left(p_1-s_0\left(p_2-s_0\left(\dotsm\left(p_{k-1}-s_0p_k\right)\dotsm\right)\right)\right)
	\intertext{Finally, substituting \(-a\) for \(s_0\), we have Horner's Method.}
			  & = p_0+a\left(p_1+a\left(p_2+a\left(\dotsm\left(p_{k-1}+ap_k\right)\dotsm\right)\right)\right) = p(a)
\end{align*}

\section{Numerical Differentiation}
The \((n+1)\)-point formula, with error term.

\begin{equation}
	f'(x_j) = \sum_{k=0}^n{f(x_k)L'_{n,k}(x_j)}+\frac{f^{(n+1)}(\xi(x_j))}{(n+1)!}\prod_{\substack{
	k=0 \\
	k \neq j}}^n{(x_j-x_k)}
\end{equation}
where \(L_{n,k}(x)\) is the \(k\)th Lagrange coefficient polynomial of \(n\) degrees.
\begin{align*}
	L_{n,k}(x)& = \prod_{\substack{
	i=0 \\
	i \neq k}}^n{\frac{x-x_i}{x_k-x_i}} \\
	L'_{n,k}(x)& = \left(\prod_{\substack{
	i=0 \\
	i \neq k}}^n{\frac{1}{x_k-x_i}}\right)\left( \sum_{\substack{
	i=0 \\
	i \neq k}}^n{\left(\prod_{\substack{
	\ell =0 \\
	\ell \neq k \\
	\ell \neq i}}^n{(x-x_\ell)}\right)}\right)
\end{align*}

When considering uniform point spacing, i.e. \(x_j = x_0+jh\) for some small \(h\),
the Lagrangian polynomial simplifies to:
\begin{align*}
	L'_{n,k}(x_j)& = \left(\prod_{\substack{
	i=0 \\
	i \neq k}}^n{\frac{1}{(x_0+kh)-(x_0+ih)}}\right)\left( \sum_{\substack{
	i=0 \\
	i \neq k}}^n{\left(\prod_{\substack{
	\ell =0 \\
	\ell \neq k \\
	\ell \neq i}}^n{((x_0+jh)-(x_0+{\ell}h)}\right)}\right) \\
	L'_{n,k}(x_j)& = \left(\prod_{\substack{
	i=0 \\
	i \neq k}}^n{\frac{1}{(k-i)h}}\right)\left( \sum_{\substack{
	i=0 \\
	i \neq k}}^n{\left(\prod_{\substack{
	\ell =0 \\
	\ell \neq k \\
	\ell \neq i}}^n{(j-\ell)h}\right)}\right) \\
	L'_{n,k}(x_j)& = \frac{1}{h}\left(\prod_{\substack{
	i=0 \\
	i \neq k}}^n{\frac{1}{k-i}}\right)\left( \sum_{\substack{
	i=0 \\
	i \neq k}}^n{\left(\prod_{\substack{
	\ell =0 \\
	\ell \neq k \\
	\ell \neq i}}^n{(j-\ell)}\right)}\right)
\end{align*}

Substituting back into our original \((n+1)\)-point formula, we have
\begin{equation}
	f'(x_j) = \frac{1}{h}\sum_{k=0}^n{f(x_k) \left(\prod_{\substack{
	i=0 \\
	i \neq k}}^n{\frac{1}{k-i}}\right)\left(\sum_{\substack{
	i=0 \\
	i \neq k}}^n{\left(\prod_{\substack{
	\ell =0 \\
	\ell \neq k \\
	\ell \neq i}}^n{(j-\ell)}\right)}\right)}+\mathcal{O}(h^n)
\end{equation}
\end{document}
